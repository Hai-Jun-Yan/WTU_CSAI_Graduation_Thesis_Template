%%%%%%%%%%%%%%%%%%%%%%%%%%%%%%%%%%%%%%%%%%%%%%%%%%%%%%%%%%
%                   声明和基本信息
%            Statement and basic information
%%%%%%%%%%%%%%%%%%%%%%%%%%%%%%%%%%%%%%%%%%%%%%%%%%%%%%%%%%
%
% 2025-11-24 v2.0
% 作者 1:杜鑫// 机构:武汉纺织大学经济学院// 联系方式:hhkaaen@hotmail.com (欢迎联系!)
% Author 1: Xin Du// Institution: School of Economics, Wuhan Textile University// Email: hhkaaen@hotmail.com (open to suggestions!)
% 作者 2:严海军// 机构:武汉纺织大学计算机与人工智能学院// 联系方式:yanhaijun66@gmail.com (欢迎联系!)
% Author 2: Yan Haijun // Institution: School of Computer and Artificial Intelligence, Wuhan Textile University // Email: yanhaijun666@gmail.com (Welcome to contact!)


% 基于原作者的架构,本次进行了优化与功能扩展,主要更新内容如下:
% 目录样式优化:重构了目录生成逻辑,修正缩进与引导线样式,使其精确匹配学位论文排版规范。
% 图表双语标题标准化:实现了图表标题的中英文自动双排。具体规范为:图片标题采用五号宋体;表格标题采用五号宋体加粗。
% 参考文献规范化:引入自定义 .bst 样式文件,严格遵循 GB/T 7714 标准进行文献著录,并针对特定需求进行了细节微调。
% 智能引用管理:配置 natbib 宏包,实现了正文引用的右上角上标显示,支持文献的自动排序与合并压缩(如 [1-3])。
% 正文字号修正:修正了原模板中正文字号偏小(五号)的问题,将其全局统一为标准的小四号宋体。
% 表格视觉微调:调整了三线表(booktabs)的线条粗细参数,提升表格的视觉美观度与打印效果。
% 页眉布局精调:校准了页眉高度与间距参数,确保版面的视觉呈现效果与 Microsoft Word 标准模板足够接近。
% 标题字体统一:将各级标题数字编号的字体由 Times New Roman 修正为黑体。
% 版面间距优化:调整了一级标题(章标题)的段前间距,修正了标题与页眉之间的垂直距离。
% 解决浮动体单独成页居中问题:修复浮动体(图或表)单独成页时的居中显示问题
% 前三页固定内容页面改为图片:前三页(中文封面,英文封面,独创性声明)内容改为使用pdf类型的矢量图元素进行填充。使用者可在word中对相关内容编辑完成之后导出pdf在此使用。

% 免责声明与许可:
% 1.非官方性质:本项目属于个人维护的开源项目,非武汉纺织大学官方发布。模板样式是根据作者个人对学校最新格式规范的理解编写的,虽已尽力确保符合要求,但无法保证 100% 无误。
% 2.使用风险自负:请在使用前务必对照学校教务处/研究生院发布的最新《学位论文撰写规范》。如果因为格式问题影响论文送审或答辩,作者不承担任何责任。
% 3.开源协议:本项目遵循 MIT License 开源协议。
% ✅ 允许:免费使用、修改、分发、私有化部署。
% ❌ 禁止:将本项目的代码或衍生作品用于商业出售(如淘宝代写、付费模板等)。

%%%%%%%%%%%%%%%%%%%%%%%%%%%%%%%%%%%%%%%%%%%%%%%%%%%%%%%%%%
%                   基本设置
%%%%%%%%%%%%%%%%%%%%%%%%%%%%%%%%%%%%%%%%%%%%%%%%%%%%%%%%%%
\documentclass[]{WTUthesis}
\usepackage{afterpage,newtxmath,newtxtext,hyperref,titlesec,tabularx,threeparttable,booktabs,graphicx,caption,cleveref,listings}
\usepackage{multirow}
\usepackage{enumitem}
\usepackage{bicaption}
\usepackage{pdfpages}
\usepackage{mathtools}
\usepackage{setspace}

%%%%%%%%%%%%%%%%%%%%%%%%%%%%%%%%%%%%%%%%%%%%%%%%%%%%%%%%%%%
% 定义图\表标题字体及样式
%%%%%%%%%%%%%%%%%%%%%%%%%%%%%%%%%%%%%%%%%%%%%%%%%%%%%%%%%%%
% 1. 定义一个新的字体样式 "wuhaosong":五号字(\zihao{5}),宋体(\songti)
\DeclareCaptionFont{wuhaosong}{\zihao{5}\songti}
% 2. 应用设置
% font=wuhaosong: 整个标题(编号+内容)都用五号宋体
% labelfont=normalfont: 确保编号(如"表1")不加粗(覆盖之前的 bf 设置)
\captionsetup[table]{labelsep=space, skip=0pt,  font={wuhaosong, bf}, labelfont=bf}
\captionsetup[figure]{labelsep=space, skip=12pt, font=wuhaosong, labelfont=normalfont}


%%%%%%%%%%%%%%%%%%%%%%%%%%%%%%%%%%%%%%%%%%%%%%%%%%%%%%%%%%%
% 果需要取消链接的带有颜色的方框,请添加以下代码:
%%%%%%%%%%%%%%%%%%%%%%%%%%%%%%%%%%%%%%%%%%%%%%%%%%%%%%%%%%%
\hypersetup{colorlinks=false,hidelinks}


%%%%%%%%%%%%%%%%%%%%%%%%%%%%%%%%%%%%%%%%%%%%%%%%%%%%%%%%%%%
% 定义表格中线的粗细
%%%%%%%%%%%%%%%%%%%%%%%%%%%%%%%%%%%%%%%%%%%%%%%%%%%%%%%%%%%
\renewcommand{\toprule}{\noalign{\vskip2pt}\specialrule{1.5pt}{0pt}{0pt}\noalign{\vskip2pt}}
\renewcommand{\midrule}{\noalign{\vskip2pt}\specialrule{0.25pt}{0pt}{0pt}\noalign{\vskip2pt}}
\renewcommand{\bottomrule}{\noalign{\vskip2pt}\specialrule{1.5pt}{0pt}{0pt}\noalign{\vskip2pt}}


%%%%%%%%%%%%%%%%%%%%%%%%%%%%%%%%%%%%%%%%%%%%%%%%%%%%%%%%%%%
% 定义公式编号格式:式(3.1)
%%%%%%%%%%%%%%%%%%%%%%%%%%%%%%%%%%%%%%%%%%%%%%%%%%%%%%%%%%%
% \songti 确保"式"字使用的是宋体,与正文一致
\newtagform{cn-eqn}{\songti\zihao{-4} 式(}{)}
\usetagform{cn-eqn}


%%%%%%%%%%%%%%%%%%%%%%%%%%%%%%%%%%%%%%%%%%%%%%%%%%%%%%%%%%%
% 自动参考文献管理设置
%%%%%%%%%%%%%%%%%%%%%%%%%%%%%%%%%%%%%%%%%%%%%%%%%%%%%%%%%%%
% 1. 加载 natbib 宏包
% square: 方括号 []
% comma: 标点用半角逗号 (满足"英文标点+半角"要求)
% sort&compress: 自动排序压缩
% super: 上标引用
\usepackage[square,comma,sort&compress,super]{natbib}
% 2. 设置参考文献列表的具体字体和行距
% \zihao{-4}: 小四号
% \songti: 宋体 (在 ctex 环境下,英文会自动匹配为 Times New Roman,只要样式文件加载了 newtxtext)
% \setstretch{1.25}: 行距 1.25 倍 (需要 setspace 宏包)
\renewcommand{\bibfont}{\zihao{-4}\songti\setstretch{1.25}}
% 3. 调整参考文献条目之间的垂直间距 (可选,如果觉得太挤或太松)
\setlength{\bibsep}{0pt}


%%%%%%%%%%%%%%%%%%%%%%%%%%%%%%%%%%%%%%%%%%%%%%%%%%%%%%%%%%%
% 定义文献引用样式
%%%%%%%%%%%%%%%%%%%%%%%%%%%%%%%%%%%%%%%%%%%%%%%%%%%%%%%%%%%
% 2. 保存原始的引用命令为 \citen
% 如果正文中需要正常大小的引用(如"文献[1]"),请使用 \citen{...}
\let\citen\cite
% 3. 重定义 \cite 命令以实现“带方括号的上标”
% \textsuperscript{} 会把内容变为上标
% \citen{} 会输出带方括号的正常引用 [1]
% 组合起来就是:上标位置的 [1]
\renewcommand{\cite}[1]{\textsuperscript{\citen{#1}}}

\let\kaiti\kaishu


%%%%%%%%%%%%%%%%%%%%%%%%%%%%%%%%%%%%%%%%%%%%%%%%%%%%%%%%%%
%                     这是分割线
%                this is a seperate line
%%%%%%%%%%%%%%%%%%%%%%%%%%%%%%%%%%%%%%%%%%%%%%%%%%%%%%%%%%
\raggedbottom

\begin{document}

% 生成中文封面
% 插入 pdf,pages={} 表示插入这个pdf的哪几个页面,也可以指定“-”表示插入所有页面
% fitpaper=true 会调整纸张大小以适应插入的PDF(可选)
\includepdf[pages={1}, fitpaper=true]{./figs/First_Three_Pages/Page_1.pdf} 

% 空页
\cleardoublepage

% 生成英文封面
\includepdf[pages={1}, fitpaper=true]{./figs/First_Three_Pages/Page_2.pdf} 

% 生成独创性声明
\includepdf[pages={1}, fitpaper=true]{./figs/First_Three_Pages/Page_3.pdf} 

% 中文摘要页面//Chinese abstract page
\cleardoublepage
% 中文摘要
\abstract{
本文研究了XXXX问题。针对XXXX挑战,提出了XXXX方法。实验结果表明,该方法在XXXX数据集上取得了优异的性能。

本文的主要贡献包括:
1. 提出了XXXX网络架构。
2. 设计了XXXX模块。
3. 在多个基准数据集上验证了方法的有效性。
}
\keywords{深度学习,语义分割}
\researchtype{应用研究}
\makeabstract


% 英文摘要页面//English abstract page
% 英文摘要
\engabstract{
This thesis studies the problem of XXXX. To address the challenge of XXXX, a XXXX method is proposed. \\
\indent Experimental results show that the proposed method achieves superior performance on XXXX datasets.\\
\indent 
}
\engkeywords{Deep Learning; Semantic Segmentation}
\engresearchtype{Application research}
\engmakeabstract


% 生成目录
\cleardoublepage
\pagestyle{tocpage}
\pagenumbering{Roman}
\tableofcontents

%%%%%%%%%%%%%%%%%%%%%%%%%%%%%%%%%%%%%%%%%%%%%%%%%%%%%%%%%%
%                     正文内容
%                   main body content
%%%%%%%%%%%%%%%%%%%%%%%%%%%%%%%%%%%%%%%%%%%%%%%%%%%%%%%%%%

\clearpage
\pagenumbering{arabic}
\pagestyle{thesis}


% ========================================================
% 第一章:一级标题示例
% ========================================================
\chapter{一级标题示例}
\textbf{一级标题(章标题)要求}:通常为\textbf{三号黑体},居中显示,段前段后有特定间距(模板已自动设置)。

这里是正文段落文字。正文排版要求如下:

(1)全文一律采用无网格、小四号宋体字,行距为1.25倍,段前段后不空行。

(2)全文所有英文和数字用小四Times New Roman字体。

(3)段落首行缩进“两个字符”。


\section{二级标题示例}
本节展示二级标题格式。

\textbf{二级标题要求}:所有二级标题为\textbf{四号黑体},左对齐, 1.25倍行距,段前、段后各设为0.5行,不接排,模板已自动设置好。

\subsection{三级标题示例}
本节展示三级标题格式。

\textbf{三级标题要求}:所有三级标题为\textbf{小四号黑体},左对齐,1.25倍行距,段前、段后各设为0.5行,模板已自动设置好。

\vspace{1em} % 增加一点垂直间距用于视觉分隔
\noindent\textbf{关于四级标题或列表项的要求:}

论文中通常不推荐使用 \texttt{\textbackslash subsubsubsection}。如果需要更细的层级,建议使用如下列表形式:

(1)这是第一点。使用括号数字排序,文字接排。

(2)这是第二点。


% ========================================================
% 第二章:图表与公式排版
% ========================================================
\chapter{第二章(图表与公式排版)}

\section{图片排版具体要求}
(1)图大小一般为高5—7cm,宽度应与原图成比例,不得变形。特殊情况下,高度可以适当放大。总而言之,一篇论文中,同类图片的大小应该一致,编排美观、整齐,比例协调。

(2)所有插图按分章编号,如第1章,第3张图为“图1.3”,注意章和图序号之间用点号标;所有插图均需有图题(图的说明),图号及图题应在图的下方\textbf{五号宋体居中标出},且图号和图题之间空一格。\textbf{模板已设置自动编号}。

(3)一幅图如有若干分图,均应编分图号,用(a),(b),(c)......按顺序编排


\section{单张图片排版示例}
\begin{figure}[h]
    \centering
    \includegraphics[width=0.7\textwidth]{figs/Slow_ Horses.png} 
    % 中文标题
    \caption{流人}
    \vspace{-9pt}
    % 英文标题
    \caption*{{Fig. \thefigure} Slow Horses}
    \label{chapter3-fig:Slow-Horses}
\end{figure}

\clearpage

\section{多张图片排版示例}
\begin{figure}[h]
    \centering
    % 第一行第一张:怪奇物语
    \begin{minipage}[b]{0.4\textwidth}
        \centering
        % 请确保图片路径正确,例如 figs/Stranger_Things.jpg
        \includegraphics[width=\textwidth]{figs/Stranger_Things.jpg}
        \centerline{\zihao{5}(a) 怪奇物语} % 子标题
    \end{minipage}
    % \hfill % 弹性间距,撑满一行
    % 第一行第二张:硅谷
    \begin{minipage}[b]{0.4\textwidth}
        \centering
        \includegraphics[width=\textwidth]{figs/Silicon_Valley.jpg}
        \centerline{\zihao{5}(b) 硅谷}
    \end{minipage}
    
    \vspace{12pt} % 两行之间的垂直间距
    
    % 第二行第一张:纸牌屋
    \begin{minipage}[b]{0.4\textwidth}
        \centering
        \includegraphics[width=\textwidth]{figs/House_Of_Cards.jpg}
        \centerline{\zihao{5}(c) 纸牌屋}
    \end{minipage}
    % \hfill
    % 第二行第二张:生活大爆炸
    \begin{minipage}[b]{0.4\textwidth}
        \centering
        \includegraphics[width=\textwidth]{figs/Big_Bang_Theory.jpg}
        \centerline{\zihao{5}(d) 生活大爆炸}
    \end{minipage}
    
    % --- 双语标题部分 ---
    \vspace{5pt}
    % 1. 中文标题
    \caption{影视剧集数据集样本示例}
    \vspace{-9pt}
    % 2. 英文标题 (包含详细说明)
    \caption*{{Fig. \thefigure} Sample images from the TV series dataset}
    
    % 标签
    \label{fig:tv-series-samples}
\end{figure}

\section{图片引用示例}
这里展示图片引用:如图 \ref{chapter3-fig:Slow-Horses} 所示,如图 \ref{fig:tv-series-samples}(a)、\ref{fig:tv-series-samples}(b)、\ref{fig:tv-series-samples}(c)和 \ref{fig:tv-series-samples}(d) 所示。

\clearpage
\section{表格排版具体要求}
(1)按章编号,如第二章第二个表为:表2.2,并加标题,表号与标题间空一格,\textbf{标题字体为五号宋体加粗},在表格上方居中排列。\textbf{模板已设置自动编号}。

(2)表格用三线表表示(特殊情况例外),与文字齐宽,\textbf{上下边线,线粗1.5磅,表内线,线粗1/4磅。}表内同一栏的数字必须上下对齐。表内不宜用“同上”“同右”“//”和类似词,一律填入具体数字或文字。表内“空白”代表未测或无此项,\textbf{表内字体为5号宋体。在三线表中可以加辅助线,以适应较复杂表格的需要。}

(3)\textbf{建议使用\texttt{booktabs}宏包进行表格的绘制,因模板中已对\texttt{\textbackslash toprule},\texttt{\textbackslash midrule},\texttt{\textbackslash bottomrule}三种水平线条的粗细进行了预设,能更方便的绘制三线表。}

\section{表格示例}
\begin{table}[h]
    \centering
    % 表内字体设置:通常要求为五号或小五号,以显得紧凑
    \zihao{5}\songti
    % 1. 中文标题
    \caption{不同方法在数据集上的性能对比}
    % 2. 英文标题
    \caption*{{Table \thetable} Performance comparison of different methods on the dataset}
    \label{tab:example-data}
    
    % 表格内容:推荐使用三线表 (booktabs宏包)
    \begin{tabular}{l|ccc|c}
    \toprule
    方法 (Method) & 指标 A & 指标 B & 指标 C & 平均值 \\
    \midrule
    Method 1     & 85.2   & 84.1   & 86.5   & 85.3 \\
    Method 2     & 88.4   & 87.9   & 89.1   & 88.5 \\
    \textbf{Ours} & \textbf{90.1} & \textbf{89.5} & \textbf{91.2} & \textbf{90.5} \\
    \bottomrule
    \end{tabular}
\end{table}

\section{表格引用示例}
这里展示表格引用:如表 \ref{tab:example-data} 所示。

% ========================================================
% 第三章:公式示例
% ========================================================
\chapter{第三章(公式示例)}

\section{数学公式要求}
\textbf{公式排版具体要求:}

(1)公式号按章编排,如式(2.3),公式居中,编号右对齐

\section{公式示例}
(1)这是一个行内公式的示例:能量守恒方程 $E = mc^2$ 嵌入在文字中。

(2)下面是一个带编号的行间公式居中对齐示例:
\begin{align}
    \mathcal{L}_{total} = \lambda_1 \mathcal{L}_{ce} + \lambda_2 \mathcal{L}_{dice} + \alpha \sum_{i=1}^{N} w_i ||x_i||^2
    \label{eq:loss}
\end{align}

(3)下面是多个带编号的行间公式居中对齐示例:
\begin{gather}
    \mathcal{L}_{ce} = - \sum_{c=1}^M y_{o,c} \log(p_{o,c}) \\
    \mathcal{L}_{dice} = 1 - \frac{2 |X \cap Y|}{|X| + |Y|} \\
    \mathcal{L}_{total} = \lambda_1 \mathcal{L}_{ce} + \lambda_2 \mathcal{L}_{dice} + \alpha \sum_{i=1}^{N} w_i ||x_i||^2
    \label{eq:loss_gathered}
\end{gather}

\section{公式引用示例}
这里展示公式引用:如公式 \ref{eq:loss} 所示。


% ========================================================
% 第四章:参考文献引用示例
% ========================================================
\chapter{第四章(参考文献引用示例)}

\section{参考文献排版要求}

(1)参考文献在整个论文中\textbf{按出现的次序列出};数量一般要求在50篇以上,其中外文参考文献应在20篇左右。

(2)\textbf{正文为小四号宋体,英文用Times New Roman体,行距为1.25倍;参考文献中的标点符号:英文标点+半角。}\

(3)该模板中列表格式严格遵循 GB/T 7714 标准(顺序编码制)。

\section{参考文献引用示例}

本模板已配置好 \texttt{natbib} 宏包以及单独的\texttt{gbt7714-numerical.bst}文件作为参考文献样式,支持自动排序和压缩。具体引用指令如下:

\subsection{单篇文献引用示例}
在正文叙述完一个观点后,使用 \texttt{\textbackslash cite\{key\}} 命令。引用标记会自动变为右上角上标。

\textbf{示例:}
引用了一个单独的文献\cite{niu1}

\subsection{多篇文献合并引用示例}
当需要同时引用多篇文献时,只需在 \texttt{\textbackslash cite} 命令中用逗号分隔 key 即可,例如\texttt{\textbackslash cite\{key1, key2, key3\}}。模板会自动将连续的编号合并为范围(如 [1-3])。

\textbf{示例:}
引用了多篇编号连续的文献\cite{niu1,niu2,niu3},引用了多篇编号不连续的文献\cite{niu1,ASPP,TPAMI2}

\subsection{文献编排样式展示}
文献\cite{niu1}为硕士学位论文

文献\cite{niu2,niu3}为中文期刊论文

文献\cite{ImageNet,TPAMI1,TPAMI2}为英文期刊论文

文献\cite{ECCV,Swin-T,ResNet-dilated}为国际学术会议论文

文献\cite{ASPP,VggNet,Vit}为arxiv预印本论文

% ========================================================
% 第五章:总结与展望
% ========================================================
\chapter{总结与展望}
\section{主要工作总结}
本文围绕$\cdots$进行了深入研究,主要完成了以下工作:

\textbf{(1)工作一$\cdots$}

$\cdots$

\textbf{(2)工作二$\cdots$}

$\cdots$

\section{未来工作展望}
未来的研究工作可以从以下几个方向展开:

\textbf{(1)方向一$\cdots$}

$\cdots$

\textbf{(2)方向二$\cdots$}

$\cdots$

\textbf{(2)方向三$\cdots$}

$\cdots$
% ========================================================
% 参考文献
% ========================================================
\clearpage
\addcontentsline{toc}{chapter}{参考文献} 
% 导入参考文献风格gbt7714-numerical.bst文件
\bibliographystyle{gbt7714-numerical}  
% 导入参考文献reference.bib文件
\bibliography{reference}


% ========================================================
% 附录
% ========================================================
\clearpage
\chapter*{附录}
\addcontentsline{toc}{chapter}{附录} 
\markboth{附录}{附录} 
\section*{附录$\mathrm{I}$ 本人在攻读硕士学位期间获得的研究成果}
% 开始自动编号列表
% label={[\arabic*]} : 设置编号格式为 [1], [2], [3]
% leftmargin=0pt     : 列表整体左边距为0 (序号贴着左边缘)
% labelsep=1em       : 【关键】调整序号和文字之间的间距
% itemsep=0.5em      : 条目之间的垂直间距
% align=left         : 标签左对齐
\begin{enumerate}[label={[\arabic*]}, leftmargin=1.5em, labelsep=0.5em, parsep=0pt, itemsep=3pt]
    \zihao{-4}\songti
    
    \item xxxxx论文(已录用,中科院SCI、CCF xxxxx)。
    
    \item xxxxx软著(软件著作权登记号:xxxxx)。
    
    \item xxxxx专利(已受理,申请号:xxxxx)。

\end{enumerate}

\section*{附录$\mathrm{II}$ 本人在攻读硕士学位期间参加的比赛与获奖}
\begin{enumerate}[label={[\arabic*]}, leftmargin=1.5em, labelsep=0.5em, parsep=0pt, itemsep=3pt]
    \zihao{-4}\songti
    
    \item 2025年获得研究生“xx奖学金”。
    
    \item 2025年获得x等“研究生学业奖学金”。
    
    \item 2024年获得xx大赛“x等奖”。

    \item 2023年获得xx大赛“x等奖”。

\end{enumerate}


% ========================================================
% 致谢
% ========================================================
\clearpage
\chapter*{致谢}
\addcontentsline{toc}{chapter}{致谢}
% 如果你的页眉显示的是“第X章”,这一步可以强制改成“致谢”
\markboth{致谢}{致谢} 
行文至此,落笔为终。回首这段求学时光,心中感慨万千。这篇论文的完成,不仅是对我研究生阶段学习成果的总结,更离不开这一路走来给予我指导、帮助与陪伴的师长、同窗及家人。

$\cdots$

$\cdots$

$\cdots$

凡是过往,皆为序章。硕士学业的结束是终点,也是新的起点。未来路漫漫,我将带着这份感恩之心,脚踏实地,砥砺前行,不负韶华。

\end{document}
	
